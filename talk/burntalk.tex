\documentclass{beamer}

\usepackage{graphicx}
\usepackage{pgffor}

\graphicspath{{./figures/}}

% \imseq{preceding text}{sequence name}{count}
\newcommand{\imseq}[3]{%
	\begin{center}%
	\foreach \n [count=\sliden] in {0,...,#3}{%
		\only<\sliden>{%
			#1%
			\includegraphics[height=0.8\textheight,width=0.8\textwidth,keepaspectratio]{#2-\n.eps}%
		}%
	}%
	\end{center}
}
% \game{sequence name}{count}
\newcommand{\game}[2]{\imseq{Turn \n\\\medskip}{#1}{#2}}

\title{Computation in the game of Life}
\author{L. Jones}

\begin{document}

\maketitle

\begin{frame}{The game of Life}{The rules}
	\begin{itemize}
		\item A zero-player game played on an infinite 2-D grid of squares
		\item Every square starts off either alive or dead
		\item A square only stays alive if exactly 2 or 3 of its neighbours are alive
		\item A dead square is revived if exactly 3 of its neighbours are alive
	\end{itemize}
\end{frame}

\begin{frame}{The game of Life}{An example}
	\game{game1}{4}
\end{frame}

\begin{frame}{The game of Life}
	\begin{itemize}
		\item Invented by John H. Conway in the 1960s
		\item Popularised by Martin Gardner in Scientific American
	\end{itemize}
\end{frame}

\begin{frame}{Some special patterns}{Still lifes}
	\game{still1}{3}
\end{frame}

\begin{frame}{Some special patterns}{Still lifes}
	\game{beehive}{3}
\end{frame}

\begin{frame}{Some special patterns}{Blinkers}
	\game{blinker}{4}
\end{frame}

\end{document}
